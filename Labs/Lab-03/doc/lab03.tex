% !TEX root = ./ICP3038_Lab_03.tex

\section*{Laboratory 3: Introduction to Templates in C++}

%This is a typical script that you will be working with at each
%laboratory session. To work with these scripts efficiently, follow the
%guidelines below.
%\begin{enumerate}
%  \item The script is not a step-by-step tutorial. It introduces the
%    problem but it is your task to solve it.
%
%  \item If you get stuck, make sure that you read \emph{Help} section
%    at the end of the document.
%
%  \item If you still have problems, ask the lecturer, demonstrator or
%    fellow students for help.
%
%  \item Try to do as much work as possible in the class, where it is
%    easier to get help.
%    
%  \item If you need help when working at home, use the Blackboard
%    discussion board.
%    
%  \item Finish all assignments at least a week before the deadline. If you get
%    stuck, you will still have one week to ask for help.
%\end{enumerate}

The aims of today's lab are:
\begin{itemize}
\item Finish Lab~2 (the \verb+StringInverter+ class);
\item Introduce the \verb+vector+ class of the Standard Template Library (STL);
\item Understand the concept of template functions. 
%\item Get familiar with template classes.
\end{itemize}

Next week, we will start working on the assignment (start writing an \verb+Image+ class).

\section*{Task -1: Last week}

Make sure you finish the \verb+StringInverter+ class from Lab~2!

\section*{Task 0: Using CMake}

Same as last week, we will use CMake to make our lives easier. 
%In the lecture, you saw yesterday that it can be extremely complicated to use the IDE to maintain the project files, etc. particularly if students use Windows, Mac, or Linux. 
%There is a ZIP file on Blackboard with 4 (almost) empty C++ source files. There is also a `CMakeLists.txt' file. 
%It is a good practice to organise your source code using directories, etc. Do not compile code in the same directory as your source code. It can get messy... 
%Extract the ZIP file, and set up the compilation environment using \verb+cmake+. 
%In the GUI, the source directory corresponds to the direction in which `CMakeLists.txt' is. 
%The binary directory is where the code will be compiled. Often, we call this directory `bin'. 
%Once the paths are set, press `Configure'. 
%The first time you run the configuration tool, you have to select a generator.
%If you want to use MSVC++ in the lab, make sure to use \emph{Visual Studio 14 2015 Win64}. 
%For Mac OS X, you may want to use the Xcode generator. 
%For Linux, you may choose Makefile. 
%Once the configuration step is over, press `Generate'. 
%The project files are now ready in the `bin' directory. 
%You can compile the code using the preferred IDE.

\section*{Task 1: STL Vector}

To store $X$ elements of the same type,
C developers would tend to use an array whereas C++ programmers would favour the \verb+vector+ class provided by the STL. 
The first solution is not practical when $X$ varies and would require dynamic allocation and recopies of the data. 

For the first task of the week, you are given one C++ file (\verb+src/TestVector.cpp+), which corresponds to a short program to get familiar with this \verb+vector+ class. At the moment it only contains some C code, your task is to improve it. 

\begin{enumerate}

\item To use the \verb+vector+ class, add:
\begin{lstlisting}
#include <vector>
\end{lstlisting}
The documentation is in \url{http://www.cplusplus.com/reference/vector/vector/}

\item In the main, create a vector of 50 random double precision floating point numbers between 0.0 and 1.0. To produce random numbers, you can use the function \verb+randd()+ that we provide. The header you need is \verb+<cstdlib>+ (for \verb+srand+, \verb+rand()+ and \verb+time(0)+). 

\item Display every element of the vector using a \verb+const_iterator+, a \verb+for+ loop, and \verb+std::cout <<+. 
Remember to visit \verb+cplusplus.com+ to access the official C++ documentation, e.g.~\url{http://www.cplusplus.com/reference/vector/vector/begin/}.

\item In the \verb+for+ loop you just wrote, replace \verb+.begin()+ by \verb+.rbegin()+, and \verb+.end()+ by \verb+.rend()+. What happened?

\item Instead of the for loop, use \verb+std::copy+ to display every element of the vector. To do so, you need to include another 2 header files:
\verb+#include <algorithm>+ for \verb+std::copy+, and \verb+#include <iterator>+ for \verb+std::ostream_iterator+. 
In \url{http://www.cplusplus.com/reference/iterator/ostream_iterator/}, you can see an example. 
\begin{lstlisting}
// ostream_iterator example
#include <iostream>     // std::cout
#include <iterator>     // std::ostream_iterator
#include <vector>       // std::vector
#include <algorithm>    // std::copy

int main () {
  std::vector<int> myvector;
  for (int i=1; i<10; ++i) myvector.push_back(i*10);

  std::ostream_iterator<int> out_it (std::cout,", ");
  std::copy ( myvector.begin(), myvector.end(), out_it );
  
  // Most people would write:
  std::copy ( myvector.begin(), 
              myvector.end(), 
              std::ostream_iterator<int>( std::cout,", " )
             );
 
  return 0;
}
\end{lstlisting}
Adapt this code to your own problem. The main difference is the template argument. 
You have a vector of \verb+double+s, in the example it is an array of \verb+int+s.

\item Remove the last $N$ elements of the vector with:
$$
	N = 10 \times \mathrm{randd()}
$$
You can use the method \verb+vector::erase+ or \verb+vector::pop_back+. 

\item Now, display the smallest and largest values contained in the vector. To do so, use \verb+std::min_element+ and \verb+std::max_element+ provided in the \verb+<algorithm>+ header. Note that these functions return an iterator. Iterators are a bit like pointers. To display the value pointed by an iterator, add a \verb+*+ before it, e.g. \verb+*ite+.

\item Finally, display the average value. 
To get the number of elements in the vector, use its \verb+size()+ method. 
To get the sum of all the elements of the vector, use \verb+std::accumulate()+ function. 
It is provided by the \verb+<numeric>+ header. 
The last parameter is:
\verb+0+ if you are summing integer numbers,
\verb+0.0+ for double-precision floating-point numbers, or 
\verb+0.0f+ for single-precision floating-point numbers.
\end{enumerate}

\section*{Task 2: Create your own template functions}

For this task, you are given two C++ files:
\begin{enumerate}
  \item \verb+include/Utils.h+, a header file with the declarations of some functions.
  \item \verb+src/TestUtils.cpp+, a test program to try your template functions.
\end{enumerate}

Modify \verb+include/Utils.h+ to convert every function as a template function. 
To implement every function, you need to write the code directly in the header. 
In \verb+src/TestUtils.cpp+, test every template function with different data types.


%\section*{Task 3: Create your own template class}
%
%This time, you have to create your own files and modify CMakeLists.txt. 
%We propose to create a template class to handle square matrices, e.g. n-by-n matrices. 
%The class should contain:
%\begin{itemize}
%\item a default constructor, 
%\item a copy constructor, 
%\item a copy operator, 
%\item \verb+operator<<+, 
%\item \verb+operator>>+,
%\item \verb+unsigned int m_size+ (default: 4) the number of rows 
%\item \verb+unsigned int getSize() const+
%\item \verb+void setSize(unsigned int)+
%\item \verb+T& get(unsigned int i, unsigned int j) const+
%\item \verb+void set(unsigned int i, unsigned int j, const T& aValue)+
%\item \verb+void setIdentity()+
%\end{itemize}
%Each method should be tested and validated in a test program. 
%Do not wait until the end to test your code. 
%A good programming strategy is to test a function just after it has been implemented. 
