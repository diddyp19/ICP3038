% !TEX root = ./ICP3038_Lab_03.tex

  
\section*{Laboratory 3: Introduction to Templates in C++}

%This is a typical script that you will be working with at each
%laboratory session. To work with these scripts efficiently, follow the
%guidelines below.
%\begin{enumerate}
%  \item The script is not a step-by-step tutorial. It introduces the
%    problem but it is your task to solve it.
%
%  \item If you get stuck, make sure that you read \emph{Help} section
%    at the end of the document.
%
%  \item If you still have problems, ask the lecturer, demonstrator or
%    fellow students for help.
%
%  \item Try to do as much work as possible in the class, where it is
%    easier to get help.
%    
%  \item If you need help when working at home, use the Blackboard
%    discussion board.
%    
%  \item Finish all assignments at least a week before the deadline. If you get
%    stuck, you will still have one week to ask for help.
%\end{enumerate}

The aims of today's lab are:
\begin{itemize}
\item Finish Lab~2 (the \verb+StringInverter+ class);
\item Introduce the \verb+vector+ class of the Standard Template Library (STL);
\item Understand the concept of template functions. 
%\item Get familiar with template classes.
\end{itemize}

% Once you are done,  start working on the assignment.
\setcounter{section}{-1}
\section{Finish Last Week's Lab}

Make sure you finish the \verb+StringInverter+ class from Lab~2!

\section{Using CMake}

Same as before, use CMake to generate the project files for XCode, MS VC++, or Makefiles. 
%In the lecture, you saw yesterday that it can be extremely complicated to use the IDE to maintain the project files, etc. particularly if students use Windows, Mac, or Linux. 
%There is a ZIP file on Blackboard with 4 (almost) empty C++ source files. There is also a `CMakeLists.txt' file. 
%It is a good practice to organise your source code using directories, etc. Do not compile code in the same directory as your source code. It can get messy... 
%Extract the ZIP file, and set up the compilation environment using \verb+cmake+. 
%In the GUI, the source directory corresponds to the direction in which `CMakeLists.txt' is. 
%The binary directory is where the code will be compiled. Often, we call this directory `bin'. 
%Once the paths are set, press `Configure'. 
%The first time you run the configuration tool, you have to select a generator.
%If you want to use MSVC++ in the lab, make sure to use \emph{Visual Studio 14 2015 Win64}. 
%For Mac OS X, you may want to use the Xcode generator. 
%For Linux, you may choose Makefile. 
%Once the configuration step is over, press `Generate'. 
%The project files are now ready in the `bin' directory. 
%You can compile the code using the preferred IDE.

\section{STL Vector}

To store $X$ elements of the same type,
C developers would tend to use an array. 
C++ programmers would favour the \verb+vector+ class provided by the STL. 
The first solution is not practical when $X$ varies and would require dynamic allocation and recopies of the data, which makes it slow and inefficient. 

For the first task of the week, you are given one C++ file (\verb+src/TestVector.cpp+), which corresponds to a short program to get familiar with this \verb+vector+ class. At the moment it only contains some C code, your task is to improve it. 

\begin{enumerate}

\item To use the \verb+vector+ class, add at the top of the file:
\begin{lstlisting}
#include <vector>
\end{lstlisting}
The documentation is in \url{http://www.cplusplus.com/reference/vector/vector/}

\item a) In the main, create an empty vector of double precision floating point numbers. See below for an example (but of integers):
\begin{lstlisting}
    // Create an empty vector of integers
    vector<int> vec;
\end{lstlisting}

 b) Double check the size of the vector, i.e.~ how many numbers are currently stored in the vector:
\begin{lstlisting}
    // Display the original size of vec
    cout << "vector size = " << vec.size() << endl;
\end{lstlisting}

c) Add 50 random numbers between 0.0 and 1.0. 
To produce random numbers, you can use the function \verb+randd()+ that the C language provides. The header file you need is \verb+<cstdlib>+ (for \verb+srand+, \verb+randd()+ and \verb+time(0)+). 
It is already included (see L.~23 of \verb+TestVector.cpp+).
The code below shows how to add 5 values into the vector:
\begin{lstlisting}
    // Push 5 values into the vector
    for(unsigned int i = 0; i < 5; ++i)
    {
        vec.push_back(i);
    }
\end{lstlisting}

d) Double check the size of the vector. Yes again, just to be sure it is 50.

\item a) Display every element of the vector. 
Vectors work a bit like C arrays. 
You can recycle the same \verb+for+ loop as L.~65 to L.~70. 
The problem is that you have to know the size of the array. 
Here, with the vector, do not hard code \verb+50+. 
Use the \verb+size()+ method of the \verb+vector+ class. 

b) Display every element of the vector again. 
This time as a C++ developer: 
using a \verb+const_iterator+, a \verb+for+ loop, and \verb+std::cout <<+. 
The loop below shows how to sequentially access all the elements of the array using an iterator:
\begin{lstlisting}
    for (vector<int>::const_iterator ite = vec.begin();
            ite != vec.end();
            ++ite)
    {
        int temp = *ite;
    }
\end{lstlisting}
The iterator (here \verb+ite+) behaves like a pointer. 
To retrieve the value pointed by the iterator, use the \verb+*+ operator. 
To move to the next element, use the \verb!++! operator
(similarly, to move to the previous element, use the \verb!--! operator). 


% Remember to visit \verb+cplusplus.com+ to access the official C++ documentation, e.g.~\url{http://www.cplusplus.com/reference/vector/vector/begin/}.

\item In the previous \verb+for+ loop, replace \verb+.begin()+ by \verb+.rbegin()+, and \verb+.end()+ by \verb+.rend()+. Also replace \verb+const_iterator+ with \verb+const_reverse_iterator+. What happened?

\item Instead of the \verb+for+ loop, use \verb+std::copy+ to display every element of the vector. To do so, you need to include another 2 header files:
\verb+#include <algorithm>+ for \verb+std::copy+, and \verb+#include <iterator>+ for \verb+std::ostream_iterator+. 
In \url{http://www.cplusplus.com/reference/iterator/ostream_iterator/}, you can see an example. 
Two further examples are given below for your convenience:
\begin{lstlisting}
// ostream_iterator example
#include <iostream>     // std::cout
#include <iterator>     // std::ostream_iterator
#include <vector>       // std::vector
#include <algorithm>    // std::copy

    ...
    ...
    std::ostream_iterator<int> out_it (std::cout,", ");
    std::copy ( vec.begin(), vec.end(), out_it );
  
    // but most people would write:
    std::copy ( vec.begin(), 
              vec.end(), 
              std::ostream_iterator<int>( std::cout,", " )
             );
    ...
    ...
\end{lstlisting}
Adapt this code to your own problem. The main difference is the template argument. 
You have a vector of \verb+double+s, in the example it is an array of \verb+int+s.

\item Remove the last $N$ elements of the vector with:
$$
	N = 10 \times \mathrm{randd()}
$$
You can use the method \verb+vector::erase+ or \verb+vector::pop_back+. 
See \url{http://www.cplusplus.com/reference/vector/vector/erase/} and \url{http://www.cplusplus.com/reference/vector/vector/pop_back/}.

\item Now, display the 

a) smallest and 

b) largest values contained in the vector. 

To do so, use \verb+std::min_element+ and \verb+std::max_element+ provided in the \verb+<algorithm>+ header. Note that these functions return an iterator. Again, iterators are a bit like pointers. To display the value pointed by an iterator, add a \verb+*+ before it, e.g. \verb+*ite+.
See \url{http://en.cppreference.com/w/cpp/algorithm/min_element} for an example.

\item Finally, display the average value. 

a) To get the sum of all the elements of the vector, use \verb+std::accumulate()+ function. 
See \url{http://en.cppreference.com/w/cpp/algorithm/accumulate} for an example. 
It is provided by the \verb+<numeric>+ header. 
The last parameter is:
\verb+0+ if you are summing integer numbers,
\verb+0.0+ for double-precision floating-point numbers, or 
\verb+0.0f+ for single-precision floating-point numbers.

b) To get the number of elements in the vector, use its \verb+size()+ method. You can't hard code 50 as the number of element because you just delete $N$ elements...
\end{enumerate}

Note that the STL vector class implements other functionalities, such as copy constructor, \verb+operator=+, etc. It can be used as an array (like in C) or as a FILO. See \url{http://www.cplusplus.com/reference/vector/vector/} for more information.


\section{Advanced C++ Programming: Create your own Template Functions}

For this task, you are given three C++ files:
\begin{enumerate}
  \item \verb+include/utils.h+, a header file with the declarations of some functions (i.e.~the functions' signature).
  \item \verb+include/utils.inl+, a header file with the definitions of these functions (i.e.~the functions' implementation).
  \item \verb+src/TestUtils.cpp+, a test program to try your template functions.
\end{enumerate}

Modify \verb+include/utils.h+ and \verb+include/utils.inl+ to convert every function as a template function. 
To implement every function, you need to write the code directly in \verb+include/utils.inl+, which is included in the header file. 
This is due to the way C++ compilers handle inline and template functions. 
In \verb+src/TestUtils.cpp+, test every template function with different data types.

For each function to convert from inline to template, there are~3 steps to follow:
\begin{enumerate}
 \item Remove \verb'inline' (in utils.inl).
      In C, we used macros as an optimized technique for compilers to reduce the execution time. In C++, we used inline functions (see \url{http://www.cplusplus.com/articles/2LywvCM9/} for an online article on this topic). The keyword \verb'inline' needs to be remove for template functions.

 \item Add \verb'template <typename T>' at the front of every function declaration (in utils.h and in utils.inl). \verb'template <typename T>' means that the corresponding function is template, and that the template type is \verb'T'. It is a sort of virtual type that the compiler replaces during the compilation with the corresponding type (\verb'int', \verb'double', \verb'char', etc.). 
 
 \item Replaces occurrences of \verb'int' by \verb'T' (in \verb'utils.inl').
\end{enumerate}
To call a template function, you can explicit the template type:
\begin{lstlisting}
    std::cout << getMinValue<int>(1, 2) << std::endl;
\end{lstlisting}
or the compiler can try to guess (if it can't, a compilation error will be generated):
\begin{lstlisting}
std::cout << getMinValue(1, 2) << std::endl;
\end{lstlisting}

Note: There are also template classes. The STL vector class is one of them.
